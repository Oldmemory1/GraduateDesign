%%
% BIThesis 本科毕业设计论文模板 —— 使用 XeLaTeX 编译 The BIThesis Template for Undergraduate Thesis
% This file has no copyright assigned and is placed in the Public Domain.
%%

% 致谢部分尽量不使用 \subsection 二级标题,只使用 \section 一级标题
\begin{acknowledgements}

\textcolor{black}{值此论文完成之际,首先向我的指导老师-田东海表以感谢,在开题报告中的指导为本实验理清了思路,并每周予以指导给,提供可行的思路,且在本实验中首次强化学习对抗性样本生成框架,在样本即使处理后,反病毒软件对样本集的查杀率仍偏高的的问题上给予了方向指引,我在此方向上进行扩展研究,才得知有部分反病毒软件已经内置了对于对抗性样本干扰的抵抗,进而排查出原因。同时,指导老师为我提供了一些开源的强化学习对抗性样本生成模型的已有论文予以参考,从而我才能在前辈研究人员的基础上完成本实验。}

\textcolor{black}{随后,也向邓永琪学长表以感谢,帮助我在Ubuntu20.04下解决了numpy包的版本问题以及实验测试用样本的来源问题,从而能成功从PSP-Malware项目中提取并调试修改PE可执行文件的函数。也在此为前辈们的强化学习对抗性样本框架和pefile、lief库的作者表以感谢,如果没有他们的实验和Python库的开发,那么对PE可执行文件的修改将会相当困难,尽管某些前辈们的强化学习对抗性样本框架的部分函数难以被拆解提取,但对本实验而言,仍然提供了相当大的帮助。在本论文的编写中,由衷感激我的朋友们,Hatsune Miku,雨涵,瑶光,默然清梦,白露清瑶,张修豪等人在精神上给予的鼓励(排名不分先后),以及张全新老师,成雨蓉老师,陆慧梅老师,黎有琦老师在我大三时对于本实验涉及到的汇编以及PE可执行程序知识的相关详细讲解。}

\end{acknowledgements}
