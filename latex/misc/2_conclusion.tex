%%
% BIThesis 本科毕业设计论文模板 —— 使用 XeLaTeX 编译 The BIThesis Template for Undergraduate Thesis
% This file has no copyright assigned and is placed in the Public Domain.
%%

\begin{conclusion}
  % 结论部分尽量不使用 \subsection 二级标题,只使用 \section 一级标题

  % 这里插入一个参考文献,仅作参考
\textcolor{black}{本实验中,笔者使用了Sarsa算法构建了一个强化学习的静态查杀规避处理框架,用于生成对一些黑盒的反病毒软件和反病毒引擎的对抗性扰动静态规避攻击样本,动作选择可以认为是多臂老虎机问题,在有限的尝试次数下探索较好的行为,尽量规避无效动作,降低修改总次数来实现规避概率最大化。实验表明,处理前后在Clam AV和火绒杀毒下检出率变化较小,这可能是因为这两款杀毒软件自带一定的抗扰动能力,能对抗本实验中对于恶意软件的针对静态检测的扰动处理,然而360杀毒的云查杀引擎在没有云上传功能下,对处理后的恶意软件样本查杀率很低。实验表明了,在采用相似检测机制的反病毒软件之间,该模型具有可迁移攻击特性,且能推测出QVM反病毒引擎的抗扰动能力较弱,说明部分基于机器学习的反病毒引擎仍然有待升级。本实验的对抗性样本生成强化学习框架在基于原有的GYM-Malware,PSP-Malware,Mab-Malware基础上,尝试了新的强化学习算法,并且降低了模块之间的耦合性,便于后续的研究人员利用已有的研究成果创建新的强化学习框架。}
\textcolor{black}{同时本实验中也存在着一些不足:}

\textcolor{black}{(1)使用的机器性能问题,为避免程序崩溃,Sarsa算法允许的最大步数上限不能调的很高,导致可能存在一些样本,这些样本再经过一些操作也能达到规避静态检测,但智能体在指定步数之内会判定扰动产生的对抗性样本无效。}

\textcolor{black}{(2)最后,本实验尚未考虑到Windows下其它恶意程序的对抗性样本生成,例如恶意shellcode,恶意JavaScript代码等。}
\end{conclusion}
