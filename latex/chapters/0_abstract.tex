%%
% BIThesis 本科毕业设计论文模板 —— 使用 XeLaTeX 编译 The BIThesis Template for Undergraduate Thesis
% This file has no copyright assigned and is placed in the Public Domain.
%%

% 中英文摘要章节
\begin{abstract}
% 中文摘要正文从这里开始

\textcolor{black}{近年来,随着勒索病毒和恶意PE可执行文件木马程序的流行,目前存在的一些反病毒软件的查杀引擎,例如360杀毒使用的基于机器学习的QVM引擎,对于恶意PE可执行文件的检测仍然有较多的漏报和误报问题,反病毒软件厂商为了解决漏报和误报问题,需要用户主动向反病毒软件的维护人员提交漏报样本和误报样本,但这无疑存在效率较低的问题。同时,部分反病毒软件过度依赖动态检测来对抗恶意软件,静态检测能力不足,难以做到在恶意软件刚出现但是没有被运行时隔离恶意软件。}

\textcolor{black}{为了提供静态对抗性恶意样本供反病毒软件训练,本文提出并实现了一种基于强化学习的Sarsa算法的静态对抗型样本生成框架,输入为PE可执行程序恶意样本集合,输出为能够规避部分反病毒软件静态检测的对抗性样本,其由三个模块组成,分别是样本筛选模块、恶意程序扫描模块、强化学习对抗性样本生成模块。样本筛选模块实现了对于收集到的样本的处理,删除不符合条件的样本,最终生成原始样本。恶意程序扫描模块实现了获取样本集处理前后的检出率,以及判定经过强化学习对抗性样本生成模块处理后的样本能否逃逸静态检测。强化学习对抗性样本生成模块由筛选后的样本和指定的行为集和参数生成能够逃逸静态检测的对抗性样本。在实验中,某个样本集经处理前后,Clam AV对其查杀率从31.3\%下降到了20.7\%,360杀毒对其的查杀率从99.1\%下降到了4.2\%。总而言之,使用本文中提出的方法生成的静态对抗样本集对反病毒软件进行训练,可以提高基于深度学习和机器学习的反病毒软件的对于恶意软件的检测能力,对恶意PE可执行文件检测的攻防对抗领域有一定的贡献。同时本模型不同于已存在的psp-mal模型,创新点在于使用了仍在更新的ClamAV反病毒软件,360杀毒等作为静态规避检测标准,而非已经过时的EMBER反病毒模型,保证了对于更流行的反病毒软件的规避。并且引入了基于文件大小变化的奖励修正,鼓励生成更小的对抗性样本。最后,操作集相比psp-mal的操作集新增了UPX加壳,图标资源增加,签名伪造等更多操作,增加了可迁移性。}

%\textcolor{blue}{摘要正文选用模板中的样式所定义的“正文”,每段落首行缩进 2 个字符;或者手动设置成每段落首行缩进 2 个汉字,字体:宋体,字号:小四,行距:固定值 22 磅,间距:段前、段后均为 0 行。阅后删除此段。}

%\textcolor{blue}{摘要是一篇具有独立性和完整性的短文,应概括而扼要地反映出本论文的主要内容。包括研究目的、研究方法、研究结果和结论等,特别要突出研究结果和结论。中文摘要力求语言精炼准确,本科生毕业设计(论文)摘要建议 300-500 字。摘要中不可出现参考文献、图、表、化学结构式、非公知公用的符号和术语。英文摘要与中文摘要的内容应一致。阅后删除此段。}

\end{abstract}

% 如需手动控制换行连字符位置,可写 aa\-bb,详见
% https://bithesis.bitnp.net/faq/hyphen.html

% 英文摘要章节
\begin{abstractEn}
% 英文摘要正文从这里开始
%In order to study……

%\textcolor{blue}{Abstract 正文设置成每段落首行缩进 2 字符,字体:Times New Roman,字号:小四,行距:固定值 22 磅,间距:段前、段后均为 0 行。阅后删除此段。}
\textcolor{black}{In recent years, with the prevalence of ransomware and malicious Portable Executable programs, most antivirus engines—such as the QVM engine, which is based on machine learning and is used by 360 Anti-Virus—still exists tremendous false negatives and false positives in malicious PE executables detection. To solve these issues, antivirus software developers require users to actively submit samples which are malicious but not detected by antivirus software or are benign but software regard them as malware to their maintenance teams, which is inefficient. Additionally, some antivirus software excessively relies on dynamic detection to identify malware, which results in deficient static detection capabilities and has difficulty in quarantining newly emerged malware before it is executed by user.}

\textcolor{black}{To provide static adversarial samples for antivirus systems training whose purpose is detecting malware more accurately, this project raises and creates a static adversarial sample generation framework which is based on reinforcement learning Sarsa algorithm. The framework which consists of three modules: sample swifting module, malware scanning module, and reinforcement learning adversarial sample generation module, needs malicious PE executable samples as input, after processing, it outputs adversarial samples which can evade static detection of some antivirus software. The sample swifting module implements the processing of samples by removing samples which do not fit this experiment. The malware scanning module calculates detection rates of sample sets before and after processing, besides,it  determinines whether samples which are processed by the reinforcement learning adversarial sample generation module can evade static detection. The reinforcement learning adversarial sample generation module generates adversarial samples which can evade static detection by using origin samples and specified behavior sets and parameters.The experiment shows the result that after processing a sample set, the detection rate by Clam AV dropped from 31.3\% to 20.7\%, while the detection rate of 360 Anti-Virus’s declined from 99.1\% to 4.2\%.Generally, training antivirus software with the static adversarial samples which are generated by the reinforcement model can enhance the malware detection capabilities of antivirus software and engines which are based on deep learning and machine learning. This experiment contributes to the defensive domain of malicious PE executable detection. Moreover, this model innovations make this model differ from the existing PSP-Mal framework.Firstly, it adopts actively updated antivirus engines such as ClamAV and 360 Anti-Virus as static evasion detection standard,which replaces the outdated EMBER antivirus model,  ensures evasion effectiveness against prevalent anti-virus software. Secondly, a reward adjustment method which is modified by the change of file size is adopted to encourage the generation of smaller adversarial samples. Secondly, the operation set has been expanded,which is different from PSP-Mal's original configuration ,contains UPX packing, icon resource modification, and signature fabrication techniques, enhances the transferability.}
\end{abstractEn}
