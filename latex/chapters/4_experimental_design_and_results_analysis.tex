%%
% BIThesis 本科毕业设计论文模板 —— 使用 XeLaTeX 编译 The BIThesis Template for Undergraduate Thesis
% This file has no copyright assigned and is placed in the Public Domain.
%%

% 第一章节
\chapter{实验设计与结果分析}

\section{实验设计}

\subsection{开发环境配置}

\textcolor{black}{实验使用如下开发环境:}

\textcolor{black}{(1)实体机操作系统及版本号:Windows10 22H2}

\textcolor{black}{(2)虚拟机软件,虚拟机操作系统:VMWare Workstation 16.1.2 build-17966106,Windows7 旗舰版
}

\textcolor{black}{(3)集成开发环境:PyCharm 2024.3.5 (Professional Edition) }

\textcolor{black}{(4)Python虚拟环境:Anaconda 24.11.3 AnacondaNavigator2.6.5 }

\textcolor{black}{(5)反病毒软件:主机安装Clam AV,建议手动更新病毒库到最新版本。虚拟机中安装火绒杀毒和360杀毒以备后续使用,本实验中不建议在主机上安装多个反病毒软件,因为可能会导致主机出现蓝屏崩溃问题,因此,其它用于后续使用的反病毒软件建议安装在虚拟机中。}

\textcolor{black}{(6)项目依赖位于项目根目录下的requirements.txt文件中,可以使用anaconda直接导入requirements.txt文件来创建Python虚拟环境。}

\textcolor{black}{(7)实体机存储:内存空间 32GB,磁盘空间 512GB}

\subsection{原始样本获取}

\textcolor{black}{首先,在开始实验前,需要提前收集足够量的恶意样本以供本实验使用,针对此问题的解决方案是从国外VirusShare的开放恶意样本库,以及部分国内安全软件论坛,例如火绒杀毒,360杀毒等官方论坛,收集并且下载恶意样本。本实验中大约使用2000个恶意样本,并将它们分为五组以备后续使用。如果是在Virus Share上收集,需要注意文件的类型,需要是PE可执行文件。}

\textcolor{black}{用于对抗性操作的无害程序,主要用于增加伪造的签名,无害段等对抗性操作,这些无害程序的获取方式,可以从国内一些软件公司的软件安装包获取,使用sigthief将签名剥落,关于sigthief的具体使用可以参考https://github.com/secretsquirrel/SigThief 的readme的相关说明,这些从良性程序剥落下的签名可以从一些官方途径下载的软件安装包可执行文件中剥落下来。用于增加无害段需要收集一些良性DLL文件,可以从Windows操作系统根目录下的一些DLL文件中获取。}

\subsection{配置VMWare虚拟机}

\textcolor{black}{需要创建实验用Windows7虚拟机并安装VMware Tools,并调整360杀毒相关设置关闭不需要的反病毒引擎,取消勾选系统修复引擎、behavioral脚本引擎,鲲鹏引擎,避免对实验结果造成干扰。同时,需要注意潜在的样本污染问题,对于360杀毒,需要关闭自动上传发现的可疑文件,取消勾选自动上传发现的可疑文件,否则可能导致有些样本被360杀毒上传到云查杀服务器数据库中,导致后续结果查杀率偏高的问题。}

\textcolor{black}{需要注意的是,安装360杀毒后需要重启虚拟机以保证360杀毒相应查杀服务开启,否则会导致实验结果出现偏差。}

\textcolor{black}{本实验不会测试360杀毒的鲲鹏引擎,是因为大多数反病毒软件,倾向于防御现有的威胁和威胁变种,而不是没有广泛传播的对抗性样本,鲲鹏引擎也如此,它多被360杀毒用于对抗目前广泛流行的恶意样本(例如银狐和某些流行的勒索病毒),而非对抗性样本或是离目前时间较久远的样本,而且鲲鹏引擎几乎没有机器学习的自学习能力,但QVM引擎具有机器学习能力,很明显,QVM引擎更适合用于本次实验的结果检验。这是因为有时需要反病毒软件对未知程序的判断更加精准以及判断速度更快,也就是尽量降低误报的可能性以及判断时间。为了减少误报率,很多反病毒软件厂商可能会使用白名单数字签名放行,或是使用文件哈希白名单来进行放行。因为例如Intel、AMD的硬件驱动程序,它们会加载驱动,如果这些硬件外设驱动程序因为释放了一些驱动文件被反病毒软件误报没有放行,将会导致某些外设驱动不能正常安装,对用户来说后果无疑是灾难性的,极有可能导致操作系统或是计算机崩溃。但这些硬件驱动程序跟某些病毒的行为很相似,某些病毒也会使用驱动来提升自己的操作权限,使自己具有更强的破坏力以及更好地针对和破坏反病毒软件,因此,对于这些驱动程序以及一些驱动安装工具,大多数反病毒软件会选择单独白名单规避此类问题。同时,360杀毒的behavioral脚本查杀引擎和系统修复引擎也需要关闭防止造成检出率偏高导致误差。}

\subsection{结果评估标准}

\textcolor{black}{首先,定义查杀率如下:}

\textcolor{black}{查杀率计算公式如式4-1:}
\begin{equation}
P_{detect\_rate}=m_{detect\_sample\_amount}/m_{total\_sample\_amount}
\end{equation}

\textcolor{black}{对于某个恶意软件集合$sample\ s_{\omega}$,某反病毒软件的查杀率被定义为经该反病毒软件扫描该恶意软件样本集合的检出恶意软件数量除以恶意软件集合中总恶意软件数量。}

\textcolor{black}{为了更好计算样本集的规避反病毒软件的查杀效果,使用该公式要求原始样本和生成的对抗性样本前后都使用同一款反病毒软件进行扫描,则查杀下降率定义如式(4-2)所示:}
\begin{equation}
\delta_{decline\_rate}=(P_{detect\_rate\_before}-P_{detect\_rate\_after})/P_{detect\_rate\_before}
\end{equation}

\textcolor{black}{语言描述为:查杀下降率=(原始样本检出率-处理后样本检出率)/原始样本检出率。}

\textcolor{black}{Virus Total检出率的定义为式(4-3)所示: }
\begin{equation}
P_{detect\_rate}=m_{detect\_engine\_amount}/m_{total\_engine\_amount}
\end{equation}

\textcolor{black}{对于某个恶意软件样本,VirusTotal检出率定义为检出引擎总数/Virus Total扫描引擎总数。}

\textcolor{black}{Virus Total检出率的下降率定义如式(4-4)所示:}
\begin{equation}
    \delta_{decline\_rate}=(P_{detect\_rate\_before}-P_{detect\_rate\_after})/P_{detect\_rate\_before}
\end{equation}

\textcolor{black}{对于某个恶意软件样本,VirusTotal检出率的下降率定义为(原始样本VirusTotal检出率-处理后样本VirusTotal检出率)/原始样本VirusTotal检出率。}