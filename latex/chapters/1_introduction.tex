%%
% BIThesis 本科毕业设计论文模板 —— 使用 XeLaTeX 编译 The BIThesis Template for Undergraduate Thesis
% This file has no copyright assigned and is placed in the Public Domain.
%%

% 第一章节
\chapter{引言}

\section{研究背景和意义}

\textcolor{black}{恶意软件是经过精心设计,被用来攻击计算机系统或计算机网络并且造成损害的软件,危害不仅限于感染型病毒和蠕虫的自身复制耗尽系统资源,破坏操作系统、后门软件开放系统端口供黑客连接从而形成僵尸网络对服务器发起分布式拒绝服务攻击、木马伪装成正常程序,实则窃取用户的敏感数据和破坏系统,为黑客提供后门、勒索病毒加密文件要求用户支付赎金等。根据统计数据,仅2022年在全球大约发生了55亿次恶意软件攻击事件\parencite{ref1}。}

\textcolor{black}{在早期,恶意软件的代码较为简单,容易被反病毒软件检测到特征值从而处理清除。但是多年以来,恶意软件的复杂性不断发展,传统的基于黑名单哈希值的恶意软件检测技术难以应对当今恶意软件复杂的混淆策略,尽管这种方法速度很快,但是难以识别新一代的恶意软件以及一些0day恶意软件\parencite{ref2}。}

\textcolor{black}{目前,为了应对恶意软件带来的威胁,许多开源以及商业杀毒软件厂商不断升级病毒库,更替杀毒软件版本。目前针对恶意软件的识别主要分为静态分析、动态分析以及混合分析(同时结合了静态分析和动态分析)这三种。而本实验研究的强化学习模型,是生成针对反病毒软件的静态特征分析的PE可执行程序对抗性样本,目的是用于反病毒软件厂商的机器学习和深度学习反病毒引擎模型训练和逆向分析工作者们学习研究,本实验未研究动态特征分析对抗性样本生成。静态特征分析包括文件Hash(如MD5 SHA256)匹配,这在很多病毒样本分析网站,例如VirusTotal,VirusSCAN等被使用,如果上传的文件和病毒库里面存在的样本的Hash值匹配,则反病毒软件判断该文件是病毒。此外,静态分析还包括资源节分析,时间戳检查,数字签名检查,函数导入表检查,特征字符串匹配,DEBUG信息检查等。}

\textcolor{black}{在过去的十几年中,学术界出现了大量利用机器学习和强化学习的恶意软件查杀模型判断恶意软件的研究成果,甚至可以运用启发式杀毒规则,检测分析软件的代码行为,来判断从未出现过的新型恶意软件\parencite{ref3},\parencite{ref4},\parencite{ref5},\parencite{ref6},\parencite{ref7},\parencite{ref8}。}

\textcolor{black}{但不幸的是,某些基于静态特征分析的反病毒软件在某些情况下有着很高的误报率,可能因为机器学习模型自身的问题,误判一些正常的软件的行为,认为这些正常软件是恶意软件,例如QVM反病毒引擎误报Microsoft Visual Studio Complier、Clang等C/C++语言编译器,以及LLVM语法分析器等软件。此外,基于机器学习的反病毒软件也存在严重的漏洞,攻击者只需要对恶意软件进行修改,甚至有时只需要增加一个资源文件改变恶意软件的Hash值,就能绕过反病毒软件的检测,恶意软件的编写者的规避技术对于反病毒软件的识别带来了极大的挑战,这促使攻防对抗的双方不断采取更先进的措施。}

\section{国内外对于对抗性样本生成研究现状}

\textcolor{black}{在过往的PE对抗性样本生成实验中,许多已有的对抗性模型生成,有些采用值函数的强化学习算法,例如使用深度学习神经网络的DQN\parencite{ref9}。}

\textcolor{black}{有些使用遗传编程算法,基于适应度进行选择,交叉,编译\parencite{ref10}。}

\textcolor{black}{有些采用蒙特卡洛搜索树方法,将对抗样本转化为路径搜索问题\parencite{ref11}。}

\textcolor{black}{这些现有的方法已经展现了一定有效性,但时序差分算法,例如Sarsa或Q-learning的研究却很稀少,因此本实验的主要研究目标是使用时序差分算法Sarsa构建强化学习对抗性样本生成模型,来生成能够规避部分静态检测的PE可执行程序对抗样本,使部分样本逃逸反病毒软件的查杀。}

\textcolor{black}{基于强化学习的方法是通过构建动作(Action),状态(State),奖励(Reward),来使智能体与环境交互,通过多次操作获取经验,挑选一个相对更好的策略来修改已有的PE恶意程序来达到逃逸反病毒软件检测。}

\textcolor{black}{在2017年,基于强化学习生成静态对抗性样本绕过黑盒的反病毒软件的GYM-Malware框架被提出。Anderson等人定义了10种不会影响恶意软件和功能的扰动操作,例如利用签名漏洞来更改恶意软件的签名,修改恶意软件的调试(Debug)信息,修改可选首部检验码(checksum),修改现有的节的名称。这些操作与DQN强化学习算法结合和病毒检测软件交互以指导选择的扰动操作\parencite{ref9},\parencite{ref13}。}

\textcolor{black}{但遗憾的是,一部分使用强化学习模型生成的对抗性样本无法在虚拟机中正常执行,有一部分恶意软件经过某些操作后被破坏了。尽管某些操作修改的部分似乎与恶意软件的代码部分无关,但还是影响到了恶意程序的功能\parencite{ref8}。}

\textcolor{black}{但GYM-Malware模型仍然对后续的研究有着很大的作用,许多后续的研究基于其工作来进行。其中封装的一部分动作(Action),被拆成了函数用于其他的研究中,例如Mab-malware。}

\textcolor{black}{有一些研究\parencite{ref14},关注了GYM-Malware中的一些造成恶意软件功能损坏无法正常执行的操作。Mab-Malware认为是Python的LIEF库导致的,并且对其进行修复,将使用LIEF库的一些对PE可执行程序的操作改为使用Python的Pefile库,以减少损坏的恶意程序数量。也有一些研究\parencite{ref15},直接删除了可能导致恶意程序遭到破坏的操作,引入随机化操作来缩小动作空间,限制强化学习智能体可执行的动作次数,鼓励强化学习智能体寻找更优秀的Action集合。}

\textcolor{black}{而由Labaca-Castro等人开展的研究\parencite{ref16},则考虑修正奖励函数,对奖励函数添加惩罚因子,为了进一步优化强化学习智能体的操作。对奖励函数的修正鼓励智能体尽可能用更少的步骤对恶意PE程序进行修改以逃逸反病毒软件的检测。}

\textcolor{black}{类似的,Gibert et al.等人使用空操作\parencite{ref17},即插入大量NOP指令来修改恶意软件,表明使用插入无意义空操作的方法对于绕过MalConv等反病毒软件一样是有效的。}

\textcolor{black}{同时,目前也存在基于梯度的对抗性样本生成,例如FGSM、 Carlini和Wagner创建的C\&W、以及deepfool模型\parencite{ref18},\parencite{ref19},\parencite{ref20}。}

\textcolor{black}{且存在一些研究能够在已知梯度信息的情况下,通过基于梯度的方法对恶意软件的字节或者外观表现形式进行修改来规避静态特征检测\parencite{ref21},\parencite{ref22},\parencite{ref23},\parencite{ref24},\parencite{ref25},\parencite{ref26}。}

\section{创新点及主要工作贡献}
\subsection{创新点}

\textcolor{black}{本实验采用的强化学习算法是基于时序查分算法的Sarsa算法。关注PE恶意程序的原因是Windows操作系统在个人电脑(PC),服务器等操作系统中占有72\%左右的份额,因此超过70\%的恶意软件将Windows操作系统作为攻击目标\parencite{ref27},\parencite{ref28}。相比已有的工作,本实验额外考虑了UPX加壳操作和sigthief造成的假数字签名,resourceHacker调用导致的图标类资源添加等行为,对静态检测的对抗性样本生成可能带来的影响。同时,使用了仍在更新的ClamAV反病毒软件,360杀毒等作为静态规避检测标准,而非已经过时的EMBER反病毒模型,保证了对于更流行的反病毒软件的规避。并且引入了基于文件大小变化的奖励修正,鼓励生成更小的对抗性样本。最后,生成的对抗性样本相比psp-mal,增加了可迁移性。}

\subsection{主要工作贡献}

\textcolor{black}{本文的主要贡献如下:}

\textcolor{black}{(1)研究了时序差分强化学习算法Sarsa对于对抗性样本的生成。}

\textcolor{black}{(2)能够生成高概率逃逸360杀毒云查杀的对抗性样本并且保证恶意程序的功能,生成的恶意程序样本即使开启了允许上传可疑文件,仍然需要经过多次扫描后上传分析才得以查杀。}

\textcolor{black}{(3)预测了未来可能出现的高危害性感染型病毒变种和结合了静态分析的挂马网站,以及部分木马下载器变种,并且建议反病毒软件厂商们加以防范,尽管动态分析技术仍然可以针对感染型病毒变种,但该类型病毒仍具有一定威胁性。这种感染型病毒变种会在感染新文件时对于新产生的病毒自带进行自动地静态检测规避处理,同时被感染的文件相较于传统的感染型病毒难以恢复。}

\textcolor{black}{(4)Action更加倾向于模块化,减少了模块之间的耦合程度,便于后续的研究人员从项目中直接提取函数,而无需修改大量内容,对于某些扰动函数只需要传入原恶意程序文件的绝对路径,和用户期望生成的文件的绝对路径,就能实现对抗型扰动操作。相比原有框架PSP-Mal\parencite{ref12},本实验项目中的扰动函数与行为(Action)相互分离,并且额外加入日志系统,扰动函数自身不需要传递过多的参数。而原框架PSP-Mal对于恶意软件的操作相对混乱,难以单独从一个行为(Action中)抽象出修改函数,对恶意软件的扰动行为和智能体之间耦合度过高,导致修改较为困难。}

\section{论文结构}
\textcolor{black}{本文第一章为引言部分,说明了本课题的研究背景和意义,介绍恶意软件检测技术的研究现状,并且简要说明了工作的内容和创新点。第二章研究了本课题相关工作的PE文件结构及强化学习算法的相关理论性知识。第三章介绍实验的各个模块的功能的设计和实现。第四章是本文的实验部分以及运行环境配置的具体过程。第五章分析结果以及提出实验中预测的未来潜在威胁恶意软件以及对于相关方向人员的一些建议。在结论中总结了本文的成果和工作的创新点,提出了工作中存在的问题以及未来的可选修改方向。}