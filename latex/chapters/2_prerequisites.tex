%%
% BIThesis 本科毕业设计论文模板 —— 使用 XeLaTeX 编译 The BIThesis Template for Undergraduate Thesis
% This file has no copyright assigned and is placed in the Public Domain.
%%

% 第一章节
\chapter{预备知识}

\section{实验预备知识}

\subsection{PE文件}

\textcolor{black}{PE文件是Windows操作系统下的可执行文件形式,包括EXE(可执行文件),DLL(动态链接库),SYS(系统文件)等类型。}

\textcolor{black}{PE文件中包含PE文件头(IMAGE\_NT\_HEADERS),其中的COFF文件头(IMAGE\_FILE\_HEADER)和可选首部(IMAGE\_OPTIONAL\_HEADER)中的有些部分是我们需要修改的目标。}]

\textcolor{black}{COFF文件头(IMAGE\_FILE\_HEADER)包含如下关键字段:}

\textcolor{black}{(1)Machine:目标CPU架构,指明了能运行这个程序的机器码,可以指明支持程序运行的机器架构是x86、x64、PowerPC、ARM等。}

\textcolor{black}{(2)NumberOfSections:指明所有节区的数量。}

\textcolor{black}{(3)TimeDateStamp:时间戳,指明了这个文件被编译生成的时间。}

\textcolor{black}{(4)SizeOfOptionalHeader:可选首部的大小。}

\textcolor{black}{(5)Characteristics:文件的类型,是动态链接库,还是可执行文件等类型。}

\textcolor{black}{可选首部(IMAGE\_OPTIONAL\_HEADER)包含如下字段,可选首部指明了当PE程序被载入内存后的一些情况:}

\textcolor{black}{(1)Magic:魔法位,包含PE信息,一定要和COFF文件头中的Machine对应,否则就会报错导致程序无法启动。}

\textcolor{black}{(2)AddressOfEntryPoint:程序入口点,即相对虚拟地址。}

\textcolor{black}{(3)ImageBase:加载机制。}

\textcolor{black}{(4)SectionAlignment:内存中节区的对齐粒度,不建议修改,否则程序可能无法启动。}

\textcolor{black}{(5)FileAlignment:文件中节区的对齐粒度,不建议修改,否则程序将无法启动。}

\textcolor{black}{(6)SizeOfImage:加载到内存后的总大小。}

\textcolor{black}{(7)SubSystem:子系统类型。}

\textcolor{black}{(8)DataDirectory:数据目录表,记录某些数据的位置及其大小。}

\textcolor{black}{PE文件中的节表(Section table)描述了每个节的属性,由多个IMAGE\_SECTION\_HEADER组成,每个条目对应一个节区,包括下列关键属性:}

\textcolor{black}{(1)Name:节区的名字。}

\textcolor{black}{(2)VirtualAddress:虚拟地址中的起始相对位置。}

\textcolor{black}{(3)SizeOfRawData:节区中数据的大小。}

\textcolor{black}{(4)PointerToRawData:节区偏移量。}

\textcolor{black}{(5)Characteristics:节区属性。}

\textcolor{black}{节区数据正常情况下应包含如下内容:}

\textcolor{black}{(1).text:代码节,存放该PE程序执行的指令。}

\textcolor{black}{(2).data:已经完成初始化的某些数据。}

\textcolor{black}{(3).rdata:只读(Read Only)的数据。}

\textcolor{black}{(4).rsrc:资源节,存放PE文件的图标等信息。}

\textcolor{black}{(5).reloc:重定位信息,可以用于动态链接库的装载过程。}

\textcolor{black}{(6).idata:import data,即导入函数信息。}

\textcolor{black}{数据目录表中包含的内容如下:}

\textcolor{black}{(1)Import Table:导入表,用于存放该PE文件依赖的动态链接库和调用的某些函数。}

\textcolor{black}{(2)Export Table:导出表,可以用于存放这个PE文件封装好的函数,多半运用于动态链接库封装函数供其余PE程序调用使用。}

\textcolor{black}{(3)Relocation Table:重定位表,修复地址偏移相关问题。}

\textcolor{black}{(4)TLS:线程存储表,与多线程程序有关,存储线程初始化数据。}

\textcolor{black}{(5)Debug Directory:存放该PE程序的调试信息。}

\subsection{强化学习}

\textcolor{black}{基本概念相关:}

\textcolor{black}{(1)智能体:在强化学习中决策,行动,学习。智能体是一个感知者,能感知并且理解当前的状态,智能体是一个决策者,能够知道在一个状态下应该采取什么行动,智能体是一个执行者,通过改变状态从而获取奖励。}

\textcolor{black}{(2)状态:描述了智能体与环境的相对状况。}

\textcolor{black}{(3)状态空间:所有状态的集合。}

\textcolor{black}{(4)动作:智能体在某一状态下能选择的操作。}

\textcolor{black}{(5)动作空间:所有动作的集合。}

\textcolor{black}{(6)状态转移:当执行一个动作时,智能体可能从一个状态转移到另一个状态的过程。}

\textcolor{black}{(7)策略:智能体在每一个状态下应该采取什么样的动作,允许分为确定性策略和随机性策略。}

\textcolor{black}{(8)奖励:作为人机交互的一个重要手段,可以设置合适的奖励来引导智能体按照我们的预期选择正确的决策,正数奖励表明我们鼓励智能体执行该行动,负数奖励表明我们不鼓励智能体执行该行动。}

\textcolor{black}{(9)回合/尝试:智能体执行一个策略与环境交互的过程中,智能体从开始状态到终止状态停止的过程被称为一个回合或尝试,一般用英文episode来表示。}

\textcolor{black}{(10)折扣因子:用于调整智能体对于近期奖励和远期奖励的重视程度,可以记作折扣因子为γ,γ在(0,1)的范围,且折扣因子的引入允许了无限长的轨迹.}

\textcolor{black}{(11)状态值:表达式如式(2-1)所示:}
\begin{equation}
V_{\pi}(s)=E[G_{t}|s_{t}=s]
\end{equation}

\textcolor{black}{状态值说明智能体在一个状态之下,最终能获取到的回报。首先,需要了解基于时序差分策略的方法可被用于估计状态值。}

\textcolor{black}{时序差分方法的表达式若式(2-2)所示:}
